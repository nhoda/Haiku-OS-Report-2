\title{\textbf{Haiku}\\OS Report Part II}
\author{COMP 3000\\ \\Troy Hildebrandt\\Nima Hoda}
\date{\today}

\documentclass{article}

\usepackage{cite}
\usepackage{url}
\usepackage{graphicx}
\usepackage{subfig}

% Page setup
% Shamelessly stolen from Pat Morin's Open Data Structures
\setlength{\textheight}{8.5in}
\setlength{\textwidth}{6in}
\setlength{\topmargin}{-0.375in}
\setlength{\oddsidemargin}{.25in}
\setlength{\evensidemargin}{.25in}
\setlength{\headheight}{0.200in}
\setlength{\headsep}{0.4in}
\setlength{\footskip}{0.500in}
\setlength{\parskip}{1.5ex}
\setlength{\parindent}{1.25cm}
%\flushbottom

\begin{document}
\maketitle

\section{Introduction}

In the last part of this multi-part report we discussed some
background information about the Haiku operating system, its
installation, its basic operation and presented a usage evaluation.
In this part we will discuss software packaging, major packages
included in the base installation and system initialization.

\section{Software Packaging}

Hello \cite{HaikuFaq}. % XXX need one citation or bibtex freaks out

\section{Major Package Versions}

\section{Initialization}

\bibliography{haiku-2}{}
\bibliographystyle{plain}

\end{document}
