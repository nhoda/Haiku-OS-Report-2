\title{\textbf{Haiku}\\OS Report Part II}
\author{COMP 3000\\ \\Troy Hildebrandt\\Nima Hoda}
\date{\today}

\documentclass{article}

\usepackage{cite}
\usepackage{url}
\usepackage{graphicx}
\usepackage{subfig}
\usepackage{booktabs}

\newcommand{\toprul}{\toprule[1.2pt]}
\newcommand{\tmidrul}{\midrule[1.2pt]}
\newcommand{\bottomrul}{\bottomrule[1.2pt]}

% Page setup
% Shamelessly stolen from Pat Morin's Open Data Structures
\setlength{\textheight}{8.5in}
\setlength{\textwidth}{6in}
\setlength{\topmargin}{-0.375in}
\setlength{\oddsidemargin}{.25in}
\setlength{\evensidemargin}{.25in}
\setlength{\headheight}{0.200in}
\setlength{\headsep}{0.4in}
\setlength{\footskip}{0.500in}
\setlength{\parskip}{1.5ex}
\setlength{\parindent}{1.25cm}
%\flushbottom

\begin{document}
\maketitle

\section{Introduction}

In the last part of this multi-part report we discussed some
background information about the Haiku operating system, its
installation, its basic operation and presented a usage evaluation.
In this part we will discuss software packaging, major packages
included in the base installation and system initialization.

\section{Software Packaging}

Despite starting life as OpenBeOS over ten years ago, Haiku is still lacking a comprehensive package management system. While a package management system is currently in the works\cite{HaikuFuturePkgMan}, there is still much to be done before it's ready for widespread deployment and use\cite{HaikuPkgTodo}. Instead, Haiku has relied on several alternate techniques and stop-gap technologies to distribute software packages.

One such stop-gap technology is the \texttt{installoptionalpackage} script included with Haiku, admitted by the developers of Haiku themselves as a ``band-aid for the lack of a proper package manager.''\cite{InstallOptionalPackage} This script provides limited functionality similar to more complete package management systems, such as the ability to list not only installed packages, but available packages as well. The benefit to this script versus other methods of obtaining software in Haiku is that the software packages available through \texttt{installoptionalpackage} are sanctioned by the developers of Haiku, with guaranteed compatibility. A list of packages can be obtained by simply typing \texttt{installoptionalpackage -l}, and then installation can be done by way of \texttt{installoptionalpackage -a <packagename>}.

The Haiku development team has attempted to maintain BeOS compatibility, which includes support for BeOS .PKG files as early as 2007.\cite{OpeningPkgFiles} At this point in time, it looks like the \textit{PackageInstaller} that comes as part of Haiku for compatibility with BeOS package files is fully functional, having personally installed Quake II for BeOS within Haiku through a .PKG file. This is great, aside from the fact that the \textit{PackageInstaller} does simply that, installs. Unlike complete package management systems, which store information on what packages are installed and allow for simple uninstallation as well, BeOS .PKG support in Haiku only allows for quick installation.

During BeOS's life span, an application known as \textit{SoftwareValet} was used for the distribution of BeOS PKG files. \textit{SoftwareValet} provided software through a centralized server known as BeDepot\cite{SoftwareValet} and provided functionality for not only registration, but also software updates of installed packages. Unfortunately, the PKG format used by SoftwareValet, and built using the \textit{PackageBuilder} tool, were proprietary and had to be reverse engineered.\cite{OpeningPkgFiles} This is likely a contributing factor in the development of an entirely new software packaging system for Haiku, and an explanation for the rudimentary support for BeOS software packages.

Another source of software for Haiku is what's known as ports. Ports of software already available on other systems can be acquired at \textit{http://ports.haiku-files.org/}. A simple script is available from this website that allows the quick installation of an application known as \textit{HaikuPorter}. This program, once installed, essentially works in much the same way as \texttt{installoptionalpackage} does, by providing a list of available ports, and providing automated download of these ports packages. .BEP (Be Ports) files are used to determine the rules of downloading, building, and installing Ports packages.\cite{BepFiles}

The process of obtaining software through \textit{HaikuPorter} is a simple one. Simply obtain the ports tree using \texttt{haikuporter -g} much like you obtain a list of available packages with something like Ubuntu's aptitude, or even the very simple \texttt{installoptionalpackage} script available in Haiku, list available ports through \texttt{haikuporter -l} and then choose a piece of software from this tree to install with \texttt{haikuporter -i <portname>}. The .BEP file provides information for automating the download, build, and installation procedures.\cite{BepFiles} A major downside to downloading software available in the ports tree is that much of the software is incomplete and incompatible with Haiku at this point in time. \textit{HaikuPorter}, much like \texttt{installoptionalpackage}, is also still not a substitute for a proper package management system. It lacks proper update, registration, and uninstallation functionality commonly associated with more comprehensive package management systems. \textit{HaikuPorter} also lacks the ability to automatically resolve dependencies, simply because its goal isn't to become as powerful as "powerful as Gentoo Portage or the FreeBSD ports system", making this a low priority.\cite{HaikuPorter}

Unfortunately in many cases, the only way to remove software on Haiku is to manually remove the directories and files associated with the application in question. The only alternative to this at this point is if an uninstallation script was provided to automate the process.\cite{AppInstallUninstall}

The goal for Haiku as far as software availability and installation is ultimately to have a proper package management system akin to that available with popular Linux distributions.\cite{HaikuFuturePkgMan} According to one of the developers responsible for developing a solution to package management for Haiku, \textit{HaikuPorter} will still play a very large role in the future. Haiku package files (.HPKG) will essentially work to resolve dependencies before invoking \textit{HaikuPorter} to complete the build and installation process.\cite{TappeOnPackages}

\section{Major Package Versions}

In this section we will discuss major third party packages included
Haiku Release 1 Alpha 3.  Third party software is included in the
Haiku base via two main mechanisms: as forks from the upstream version
and as ports using the haiku ports framework \cite{HaikuR1A3Src}.  The
forks themselves fall into two categories, those tracking an upstream
version, and those maintained by Haiku developers more or less
independently.  In some cases the package is maintained independently
but bits and pieces of newer versions of the package have been merged
in on a piecemeal basis.

The following table summarizes information about the packages we'll be
discussing in this section.

\begin{tabular}{l l l l}
\toprul
\textit{Package} & \textit{Fork} & \textit{Version} & \textit{Latest Stable Release} \\
\tmidrul
NewOS Kernel & Yes & \string~2002 (snapshot)\footnotemark[1] & 20050620\footnotemark[16] \\
\midrule
GNU libc & Yes & \string~2.2.5-2.3.5 (2005)\footnotemark[2] & 2.14.1 (2011-10-07) \\
\midrule
GNU bash & Yes & 4.0.35 (2009-10-24)\footnotemark[3] & 4.2.10 (2011-05-03) \\
\midrule
GNU coreutils & Yes & 8.4 (2010-01-13)\footnotemark[4] & 8.14 (2011-10-12) \\
\midrule
GNU tar & No & 1.25 (2010-11-07)\footnotemark[5] & 1.26 (2011-03-13) \\
\midrule
GNU sed & No & 4.2.1 (2009-06-27)\footnotemark[6] & 4.2.1 (2009-06-27) \\
\midrule
GNU grep & Yes & 2.5.1 (2002-03-26)\footnotemark[7] & 2.9 (2011-06-21) \\
\midrule
OpenSSH & No & 5.8p2 (2011-09-06)\footnotemark[8] (2011-05-02) & 5.9p1 \\
\midrule
p7zip & No & 9.13 (2010-05-30)\footnotemark[9] & 9.20.1 (2011-03-16) \\
\midrule
GNU make & No & 3.82 (2010-07-28)\footnotemark[10] & 3.82 (2010-07-28) \\
\midrule
bzip2 & No & 1.0.6 (2010-09-20)\footnotemark[11] & 1.0.6 (2010-09-20) \\
\midrule
WebKit & Yes & r57734 (2010-04-09)\footnotemark[12] & r100096 (2011-11-13)\footnotemark[16] \\
\midrule
Perforce Jam & Yes & 2.5 (2003-04)\footnotemark[13] & 2.5 (2003-04) \\
\midrule
gcc2 & Yes & 2.95.3 (2001-03-16)\footnotemark[14] & 4.6.2 (2011-10-26) \\
\midrule
gcc4 & Yes & 4.4.4 (2010-04-29)\footnotemark[15] & 4.6.2 (2011-10-26) \\
\bottomrul
\end{tabular}

\footnotetext[1]{\url{http://newos.org/snapshots/2002/}}
\footnotetext[2]{\url{http://ftp.gnu.org/gnu/libc/}}
\footnotetext[3]{\url{ftp://ftp.gnu.org/gnu/bash/bash-4.0.tar.gz}, \url{ftp://ftp.gnu.org/gnu/bash/bash-4.0-patches/bash40-035}}
\footnotetext[4]{\url{http://ftp.gnu.org/gnu/coreutils/coreutils-8.4.tar.gz}}
\footnotetext[5]{\url{http://ftp.gnu.org/gnu/tar/tar-1.25.tar.bz2}}
\footnotetext[6]{\url{http://ftp.gnu.org/gnu/sed/sed-4.2.1.tar.gz}}
\footnotetext[7]{\url{http://ftp.gnu.org/gnu/grep/grep-2.5.1.tar.gz}}
\footnotetext[8]{\url{http://ftp.openbsd.org/pub/OpenBSD/OpenSSH/portable/openssh-5.8p2.tar.gz}}
\footnotetext[9]{\url{http://downloads.sourceforge.net/project/p7zip/p7zip/9.13/p7zip_9.13_src_all.tar.bz2}}
\footnotetext[10]{\url{http://ftp.gnu.org/pub/gnu/make/make-3.82.tar.bz2}}
\footnotetext[11]{\url{http://www.bzip.org/1.0.6/bzip2-1.0.6.tar.gz}}
\footnotetext[12]{\texttt{svn checkout} \url{http://svn.webkit.org/repository/webkit/trunk@r57734} \texttt{WebKit}}
\footnotetext[13]{\url{ftp://ftp.perforce.com/jam/src}}
\footnotetext[14]{\url{ftp://ftp.gnu.org/gnu/gcc/gcc-2.95.3/}}
\footnotetext[15]{\url{ftp://ftp.gnu.org/gnu/gcc/gcc-4.4.4/}}
\footnotetext[16]{This is a snapshot or revision number since this software package does not issue stable releases}


\paragraph{Kernel}
The Haiku kernel is a fork of the NewOS kernel, which was written by a
former BeOS engineer Travis Geiselbrecht \cite{HaikuWiki}.  A search
for the copyright notices in the Haiku kernel sources suggests the
fork was made sometime in 2002, though the exact source is difficult
to determine, especially since NewOS never made any regular releases.

The most recent NewOS snapshots were released in 2005, however, the
Haiku commit logs don't indicate any merging of NewOS sources into the
Haiku repositories since then but rather indicate significant local
modifications \cite{HaikuKernelCommitLogs}.

It's difficult to say why exactly this kernel was chosen for the
project.  A review of the Haiku developer mailing lists didn't yield
any clues as to the rationale for the choice.  It's possible that the
involvement of a former BeOS engineer made both the NewOS kernel a
good fit for a BeOS clone as well as made BeOS enthusiasts aware of
NewOS.

\paragraph{C Library}
Haiku includes a fork of the GNU C Library in its libroot library, as
did BeOS \cite{GlibCWiki}.  It appears from the commit logs that much
of the 2.2.5 version of glibc was imported into the Haiku
fork \cite{HaikuLibrootGlibcOld} but that parts of the 2.3.x versions
were also brought in on a piecemeal
basis \cite{HaikuLibrootGlibcRecent} so it's difficult to specify
exactly which upstream version Haiku's glibc is derived from.  The
logs also indicate much local modification of glibc.

Of course, a C library was necessary for Haiku to reimpliment the BeOS
API.  The GNU libc was most likely chosen over a BSD libc because BeOS
itself included GNU libc \cite{GlibCWiki}.

\paragraph{GUI Foundation}
Haiku provides its own GUI Foundation in \texttt{app\_server} and so
does not require a 3rd party package such as x.org.

\paragraph{GUI Toolkits}
Haiku also provides its own GUI Toolkit in its reimplimentation of the
BeOS APIs.

\paragraph{Shells}
Like BeOS, Haiku includes a terminal application which provides a
command-line interface with GNU bash.  A recursive diff between
Haiku's bash sources and the ones from its upstream source show
relatively few changes, most of which appear to have more to do with
porting the shell than with modifying it (i.e. preprocessor defines).

\paragraph{Utilities}
In order to support the command-line environment, Haiku includes a
variety of command-line utilities from various third party software
packages, including GNU coreutils (cat, cp, echo, ls, mv, mkdir, true,
etc.), GNU tar, GNU sed, GNU grep, OpenSSH, p7zip, GNU make and bzip2.
Of these, recursive diff's and patch files \cite{HaikuR1A3Src}
indicate that only coreutils had any sizable modifications.  There,
most of the changes had to do with adding support for copying file
meta data known as ``attribututes,'' a special feature of the BFS
filesystem \cite{BFSWiki}.  The other utilities were either unmodified
or modified just for porting purposes with the addition of
preprocessor defines.

\paragraph{Software Packaging}
Haiku does not include any 3rd party package management systems.  A
ports tree and package management system being developed for Haiku are
described above.

\paragraph{Web Browser}
Haiku includes the WebKit-based WebPositive as its default web
browser \cite{WebPositiveWiki}.  Though WebPositive is developed
specifically for Haiku, it's sources are kept in a separate
repository \cite{WebPositiveTrac} that includes a fork of the WebKit
engine.  A recursive diff of the WebPositive WebKit and the WebKit
revision it's based on results in a more than 25000 line patch,
indicating heavy modifications to the upstream source.

WebPositive was developed to replace the old Firefox2-based
BeZillaBrowser \cite{WebPositiveWiki}.  It's unclear why Haiku
developers chose the WebKit engine as opposed to a newer version of
the Mozilla engine.  It may be because of WebKit's growing market
share and its backing by the major corporations Google and Apple.

\paragraph{Email}
Haiku includes its own e-mail application based on the original open
sourced BeOS Mail application \cite{BeMailOpenSourced} which, based on
the LICENSE file, Be, Inc. released in 2001 \cite{BeMailLICENSE}.  The
Haiku version has been modified to support multiple
accounts \cite{BeMailOpenSourced}.  This software was included in
Haiku to help fulfil Haiku's goal of reimplimenting the BeOS
experience.

\paragraph{Build Tools}
In order to maintain both backwards compatibility with BeOS R5
applications and permit the porting of modern packages, Haiku includes
both gcc 2.95.3 and gcc 4.4.4 \cite{GCCHybrids}.  gcc is an essential
part of Haiku allowing the OS to be self-hosting (build itself) and to
build third party software packages.  The gcc 4.4.4 in Haiku is
relatively unmodified.  A recursive diff with the upstream 4.4.4
produces a 1600 line patch.  Haiku's gcc 2.95.3, however, is very
heavily modified, producing a 45,000 line patch when compared to its
upstream source.

\section{Initialization}

Haiku initializes itself through a Bash shell script called \texttt{Bootscript}, which the kernel calls upon directly to initialize applications and services when Haiku starts. \texttt{Bootscript} also calls upon a \texttt{SetupEnvironment} script that initializes important environment variables. Below is a short list of just a few of the applications and services that start when Haiku is initialized, in order of initialization.

\paragraph{registrar}
A service in charge of maintaining information about MIME types and
identifying the types of files in order to fill out their type
attributes. \cite{RegistrarInfo}

\paragraph{net\_server}
A service responsible for configuring network
interfaces. \cite{NetServerSource}

\paragraph{app\_server}
``At the heart, it manages multiple applications simultaneously using the display device as a shared resource.''\cite{AppServer}

\paragraph{syslog\_daemon}
A service providing the syslog interface defined in POSIX upon which
the BeOS logging capabilities are built in Haiku \cite{SyslogInfo}.

\paragraph{mount\_server}
A service that listens for system messages indicating new media has been detected and automatically mounts associated filesystems.\cite{AutoMounter}

\paragraph{Tracker}
An application that provides one-stop access to file system exploration. Clicking the \textit{Tracker} button brings up a list of all open windows currently viewing any mounted file systems.\cite{Tracker}

\paragraph{Deskbar}
Similar to the Windows task bar, its \textit{Start} bar is instead a button with the Haiku feather. It behaves exactly as one would expect, bringing up menus that provide access to applications, shutdown/sleep functionality, recent documents, system preferences, and more.\cite{Deskbar}

This was all discovered through a fortunate accident, finding the \texttt{Bootscript} in its logical location, \texttt{/boot/system/boot/}. Several scripts are found in this directory, and a \texttt{more} on Bootscript revealed its importance in Haiku's boot process. A \texttt{grep} on the kernel source files confirmed that \texttt{Bootscript} was indeed solely responsible for the order of application and service initialization in Haiku.

\bibliography{haiku-2}{}
\bibliographystyle{plain}

\end{document}
